\chapter{Conclusion}
\label{ch:conclusion}

\section{Difficultés rencontrées}
Ce travail étant le premier de cette ampleur que j'effectue, j'ai été
confronté à quelques difficultés que je souhaite énumérer ici.

\subsection{Problèmes liés au Covid-19}
Depuis mars 2020, la pandémie SARS-CoV-2 a complétement changé la façon
de travailler et ce jusqu'à la fin de mon travail de Bachelor en juillet 2020.

Ces changements ont apportés quelques complications:

\subsubsection{Complication des tests}
À cause de l'interdiction de se rassembler, et également pour les raisons légales et morales (cf \ref{ch:etudelegislation} et \ref{ch:etudemoralite}),
il ne m'a pas été possible de tester mon produit en situation réelle, menant à avoir quelques réserves quant à son efficacité effective.

Le module de simulation développé pour pallier ce problème a toutefois permis de tester d'autres aspects qui ne l'auraient pas été en condition réelle, comme un très grand nombre
de personnes, ou une simulation de plusieurs dizaines d'heures. 

\subsubsection{Cadre de travail et motivation}
Étant confiné pendant la majeure partie de mon travail, il a parfois été compliqué
d'entretenir une motivation sur le long terme à cause de l'isolement et du climat parfois anxiogène.

\subsection{Le matériel}

\subsubsection{PiCaméra défectueuse}
Après quelques semaines d'utilisation, un des modules PiCamera a arrêté de fonctionner.
Il semblerait que cela soit un disfonctionnement matériel. Cet incident m'a fait perdre du temps à cause du dépannage et de la nouvelle commande.

\section{Améliorations futures}
Dans le futur, ces quelques idées pourraient être implémentées pour étendre les capacités du prototype ou l'améliorer.
\begin{itemize}
    \item Ajouter un serveur WSGI devant l'API et le front-end afin de rendre le produit plus "production-friendly"
    \item Améliorer l'algorithme PP2I pour en diminuer la complexité algorithmique
    \item Analyser les implémentations actuelles de la randomisation d'adresses MAC afin de trouver de nouvelles contre-mesures
    \item Ajouter un module permettant l'attaque active (e.g~\cite{WIFIKARMA}) sur les clients raspberry (e.g une autre interface pouvant créer des evil twins)
    \item Développer un mode "privacy-friendly" qui ne permettraient aux opérateurs du système de n'effectuer que des statistiques de masse sans connaître d'informations individuelles
\end{itemize}

\section{Retour personnel}
Ces quelques mois de travail ont été très enrichissants.
Tout d'abord, il s'agit du travail le plus personnel qui m'a été donné de faire.
Habitué à des travaux de groupe, j'ai pu goûter à une nouvelle forme d'autonomie, ce qui m'a permis de fournir
un projet dont je suis fier et dont je maîtrise chaque partie. 

Je suis satisfait des résultats obtenus lors des diverses étapes du projet, notamment ceux concernant l'algorithme PP2I que j'ai conçu.
J'ai toutefois été un peu déçu de ne pas avoir pu m'attaquer à la randomisation des adresses MAC, par manque de temps et de ressources.

Je suis également surpris de la facilité avec laquelle il a été possible d'inclure des services de reconnaissance faciale au projet. Je n'imaginais pas
que l'état de l'art permettait cela avec mes connaissances actuelles.

\section{Remerciements}

Je suis infiniment reconnaissant envers toutes les personnes ayant contribué à la réussite de mes
études, et de ce travail de Bachelor en particulier.

En tout premier lieu, je tiens à remercier mon responsable de travail: Monsieur Abraham Rubinstein.
De par ses précieux conseils, son suivi régulier ainsi que son optimisme lors d'une période compliquée pour tous, il m'a permis
d'affirmer au mieux mes compétences, et ainsi de rendre un travail dont je suis fier.

Je souhaite également adresser ma gratitude à tous les professeurs qui ont investi de leur temps et leurs efforts pour
guider mes camarades et moi vers l'excellence académique. J'ai énormément appris pendant ce Bachelor grâce à leur travail exceptionnel.

Pour leur patience, et tout ce qu'ils m'ont donné, je renouvelle dans ce document, le respect, l'amour et la gratitude que j'éprouve pour mes parents, Corinne et Philippe Polier.
Toujours encourageants, c'est leur soutien indéfectible qui fait de moi l'étudiant et la personne que je suis aujourd'hui.

Pour conclure, je remercie et adresse mes voeux à mes camarades. 
Travailler avec eux a été une source inépuisable de motivation et je leur souhaite à tous la réussite professionelle.
Certains d'entre eux ont activement collaborés à mon travail: Notamment Julien Quartier, qui m'a offert un peu de son temps ainsi que ses connaissances
en m'imprimant le support des modules PiCamera.



