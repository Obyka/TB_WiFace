\chapter{Étude de marché}
\label{ch:etude-de-marche}

\section{Nano-ordinateurs}

Afin de pouvoir faire fonctionner le prototype développé, il est nécessaire d’utiliser des nano-ordinateurs (un par
caméra), ce qui offre le meilleur compromis entre puissance de calcul, mobilité et prix d’acquisition.

Voici les critères permettant de choisir le modèle à utiliser :
\begin{itemize}
\item Installation libre d’une distribution Linux compatible avec les outils à utiliser (librairies, langages, drivers)
\item Compatibilité hardware avec les cartes réseaux 802.11 et les caméras choisies à la section \ref{sec:choix_camera}
\item Puissance processeur et RAM suffisantes pour exécuter les tâches les plus lourdes du projet (notamment le traitement à l'aide d'OpenCV)
\item Prix adéquat et compatible avec le budget associé au travail de Bachelor
\item Documentation fournie et/ou communauté active
\end{itemize}

Le tableau~\ref{tab:etude-marche-comparatif} résume le comparatif.

\begin{table}[H]
    \resizebox{\textwidth}{!}{%
    \begin{tabular}{@{}p{5cm}p{5cm}p{5cm}p{5cm}@{}}
    \toprule
     & \textbf{NanoPC-T4}\cite{NANOWIKIT4} & \textbf{Raspberry Pi 4}\cite{MANRASP} & \textbf{ODROID-XU4}\cite{MANXU4} \\ \midrule
    \textbf{Installation Libre} & Partielle & Oui & Partielle \\[0.5cm]
    \textbf{Compatibilité caméra} & Module caméra existant & Module caméra existant & Caméra USB \\[0.5cm]
    \textbf{Compatibilité antenne 802.11} & Oui & Oui & Oui \\[1cm]
    \textbf{Spécifications du processeur} & Dual-Core Cortex-A72 (up to 2.0GHz) + Quad-Core Cortex-A53 (up to 1.5GHz) & Broadcom BCM2711, Quad-Core Cortex-A72 (ARM v8) 64-bit SoC @1.5GHz & Samsung Exynos5 Octa ARM Cortex™-A15 Quad 2Ghz and Cortex™-A7 Quad1.3GHz CPUs15 \\
    \textbf{RAM} & Dual-Channel 4GB LPDDR3-1866 & 1GB, 2GB or 4GB LPDDR4-3200 SDRAM & 2GB LPDDR3 RAM at 933MHz \\[1cm]
    \textbf{Prix sans extension} & $\sim$150 \$ & $\sim$65 \$ (Modèle 4GB) & $\sim$80 \$ \\[0.5cm]
    \textbf{Qualité de la documentation} & Un wiki & Communauté très active et beaucoup de documentation & Quelques user guides, un wiki \\ \bottomrule
    \end{tabular}%
    }
    \caption{\label{tab:etude-marche-comparatif}Comparatif des nano-ordinateurs}
    \end{table}

Au vu de la qualité de la documentation, du prix raisonnable du modèle possédant 4GB et de tous les autres critères
remplis, la meilleure alternative pour le projet semble être la \textbf{Raspberry Pi 4}.

\section{Caméra pour la reconnaissance faciale}
\label{sec:choix_camera}

Afin de pouvoir enregistrer des images, pour les traiter et ainsi reconnaître des visages, une caméra devra être
associée à la raspberry pi. Ce use-case étant fréquent, un module caméra officiel\cite{MANRASPCAM} est proposé à la vente. Le tableau~\ref{tab:etude-marche-picamera}
énumère quelques spécifications :

\begin{table}[H]
    \resizebox{\textwidth}{!}{%
    \begin{tabular}{@{}llll@{}}
    \toprule
    \textbf{Prix} & \textbf{Résolution} & \textbf{Modes vidéo} & \textbf{Driver Linux} \\ \midrule
    Environ 25 dollars & 8 Megapixels & \begin{tabular}[c]{@{}l@{}}1080p30, 720p60, \\ 640×480p60/90\end{tabular} & V4L2 driver
    \end{tabular}%
    }
    \caption{\label{tab:etude-marche-picamera}Spécification du module PiCaméra}
\end{table}

Au vu des nombreux projets open-source utilisant ce module pour faire de la reconnaissance faciale, et du fait qu’il
soit un produit agréé, ce dernier sera choisi pour le projet. Il existe également une version V1 de ce module,
proposant des performences moindres, pour un prix sensiblement identique.

\section{Antenne 802.11}

Afin de pouvoir sniffer les probe requests, il faudra une carte WiFi capable de se mettre en mode « Monitor » (pour
recevoir les paquets qui ne sont pas directement adressé à l’adresse de la Raspberry). Comme il sera nécessaire
que la raspberry soit également connectée à internet pour envoyer des données, une antenne supplémentaire sera
utilisée. L’école possède déjà des antennes AWUS036H compatibles, qui permettront d’effectuer le sniffing.



