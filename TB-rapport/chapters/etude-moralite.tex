\chapter{Éthique et moralité des systèmes de surveillance}
\label{ch:etudemoralite}

Au delà de l’aspect législatif abordé dans le chapitre précédent, il est essentiel pour un ingénieur de mesurer ses
responsabilités morales quant aux produits qu’il aide à créer. N’étant pas convenablement formé sur ce sujet, je
me permettrai de me baser sur le travail d’autrui et de faire le lien avec le projet «WiFace».

\section{La reconnaissance faciale, une technologie déshumanisante}
Dès le début des recherches, il est apparu clairement que la reconnaissance faciale allait être l’aspect soulevant le
plus de questionements moraux dans mon travail, bien plus que l’association d’une identité et d’un terminal.
Pourquoi est-ce le cas ? Les deux situations permettent pourant de caractériser une personne de manière unique.

Le Prof. Brey l’explique en qualifiant le procédé de reconnaissance faciale de «déshumanisant». En effet, pouvoir
encoder le visage (ses features) d’une personne – et donc une partie de son unicité - sur quelques bytes semble
dérangeant pour beaucoup. De plus, un visage est partie intégrante de l’identité d’un individu, alors qu’un terminal
n’est finalement qu’une possession temporaire.

Une autre problèmatique inhérente à cette technologie est celle de l’atteinte à la vie privée et à la confidentialité.
Un système permettant d’identifier, tracer, journaliser et incriminer un individu dans l’espace publique ou privé
nuit gravement aux droits des individus tels que mentionnés dans l’Article 8 de la Convention européenne des droits
de l’homme.

Cette dernière proclame le droit de toute personne au respect « de sa vie privée et familiale, de son domicile et de
sa correspondance », concrétisant ainsi nos questionnements.

\section{L’opposition entre sécurité et confidentialité}
Malgré tout, une part de l’opinion publique approuve l’implémentation de tels systèmes de reconnaissance.
Comme argument principal, nous retrouvons souvent l’augmentation de la sécurité.

L’exemple des passeports bioémtriques est parlant: certains acceptent volontiers son utilisation car les avantages
(lutte contre le terrorisme, l’usurpation d’identité, augmentation de la facilité de déplacement) semblent
supérieurs aux désavantages (processus déshumanisant, atteinte à la liberté) mais il s’agit d’un cadre très précis, et
ces même personnes ne seraient pas forcément d’accord de trouver de pareils systèmes dans d’autres
circonstances (dans l’espace publique, dans des établissements privés). Cela nous amène donc à la première dérive
: le «Function creep» (dérapage fonctionnel).

\section{Le function creep}
Ce terme, emprunté de l’auteur John Woodward est le phénomème par lequel une technologie développée dans
un certain but outrepasse ces derniers et élargit son champs d’utilisation. Cela peut être dû à un usage abusif ou
par des changements législatifs la concernant.

L’étude susmentionnée prend l’exemple des Smart CCTV au début des années 2000 (caméras de surveillance
couplées à un système de reconnaissance faciale.) Ces dernières ont été testées par la police afin de retrouver les
personnes disparues et identifier des criminels inscrit dans une base de données.

Cette technologie étant très versatile, il est facile d’imaginer d’autres use-case pour la même implémentation de
ce système. Par exemple, l’utilisation abusive pour le bénéfice personnel d’un agent de police (surveillance de ces
proches, utilisation illégitime des images capturées). Le système initiale semblait alors moralement acceptable mais
son dérapage fonctionnel l’a rendu intolérable, pourtant ce dernier n’est pas forcément détectable par les sujets
de cette technologie (le grand public).

\section{La fiabilité et les erreurs}
Comme nous l’avons expliqué, il existe déjà des implémentations servant des systèmes très critiques, pouvant
mener à des arrestations et d’autres conséquences importantes pour ses sujets. Or, bien qu’elle s’améliore avec le temps, la reconnaissance facial est sujette à des erreurs. Dans les systèmes de surveillance, cela amène à une
méfiance générale, et à un sentiment d’insécurité. 

Il existe deux types d’erreurs ayant des conséquences différentes:
\begin{enumerate}
\item Le faux négatif : Une personne devant être identifiée ne l’est pas, le système ne remplit pas on objectif et
donc ses désavantages outrepassent ses avantages
\item Le faux positif: Une personne ne devant pas être identifiée est mal reconnue et prise pour quelqu’un
d’autre, le système incrimine une personne innocente / non-concernée, cela décridibilise le système et
amène à des conséquences néfastes
\end{enumerate}

La deuxième catégorie d’erreur est plus critique que la première, c’est pourquoi la majeur partie des
implémentations choisisse un seuil minimal de confiance très haut même si cela mène à un nombre supérieur de
faux négatifs.

Pour ne rien arranger, il a été montré dans une étude effectuée par le NIST (National Institute of Standards and
Technology) que la plupart des meilleures solutions commercialisées de reconnaissance faciale possédent un biais
lié à l’origine du sujet. En effet, le taux d’erreur est 10 à 100 fois supérieurs sur les sujet Afro-Américains et
asiatiques par rapport aux sujets caucasiens.

Un tel biais a le potentiel d’être un facteur d’augmentation de la tension sociale et pourrait impacter la législation
en criminalisant et discriminant injustement certains groupes d’individu par rapport à d’autres.

\section{La vie privée}
Le dérapage fonctionnel et la possibilité d’erreurs n’exprime pas totalement en quoi de telles technologies portent
atteinte à la vie privée.

De manière empirique, nous pouvons remarque que certaines personnes pensent qu’il est incompatible de lié
espace publique et vie privée, puisque le premier ne pourraient pas exister dans le deuxième, et donc que la
surveillance n’est pas un problème. Pourtant, dans son essai de 1998, l’auteure Helen Nissenbaum argumente le
contraire. Elle affirme que la récole et le stockage d’informations à l’aide de dispositifs automatisés amène souvent
à la violation de la vie privée.

Pour appuyer ses propos, elle se base sur deux prémises:
Premièrement, en sachant que la plupart des gens se sentent surpris ou mécontents quand ils apprennet que leur
données personnelles ont été collectées à leur insu, même dans un espace publique, elle prouve que ces personnes
s’attendaient à un certain respect de leur vie privée, même dans des lieux de vie communs.

Deuxième, elle affirme que l’automatisation de la récolte des données est bien plus problématique qu’une
surveillance «manuelle» dans l’espace publique, et ce à cause de deux pratiques rendues possibles par la
technologie. La première pratique rendue possible est le changement de contexte de l’information une fois
enregistrée (via la vente de données recoltées par exemple). Or l’être humain donne beaucoup d’importance au
contexte quand il délivre une information volontairement (on parle de ses finances à son banquier, de ses
problèmes de santé à son médecin, mais pas inversemment). La journalisation des données amène alors à de
potentielles atteintes à la vie privée. La deuxième pratique est « l’aggrégation des données» ou le Big Data.
L’observation individuelle de certaines actions ou habitudes n’a pas vraiment de conséquences, mais lorsque ces
systèmes sont capables de mettre en lien beaucoup d’information et d’en tirer des corrélation (par exemple à l’aide
du machine learning) de nouvelles informations peuvent être déduites sans le consentement explicite du sujet,
menant à une autre forme d’atteinte.

\section{Conclusion}
Nous avons examinés certaines implications morales d’un système de surveillance et de récolte de données, en
particulier sur la technologie de la reconnaissance faciale. Parmi les problèmes principaux, nous avons identifié:
l’atteinte à la vie privée, de dérapage fonctionnel, et la potentialité d’erreur.

Au moment de sa conception, WiFace ne dispose pas d’un use case établi. Il s’agit d’un outil, sans cadre prédéfini.
Ce constat nous permet d’affirmer qu’il est donc fortement sujet au dérapage fonctionnel, puisque n’importe quelle
personne pourrait facilement déployer cette solution à bas coût pour son bénéfice personnel.

À la fin de ce travail, WiFace doit être amené au stade de prototype et non de projet fini. Ce statut atteste de son
manque de fiabilité potentiel. Il est donc probable ques on utilisation amène à des erreurs, ce qui pose un autre
problème éthique.

L’atteinte à la vie privée semble inhérente à tout système de collecte de données automatisée, WiFace ne faisant
pas exception. En permettant de lier un terminal à un utilisateur, nous pouvons facilement imaginer des situations
où cette information pourrait être sortie de son contexte (publicité ciblée, statistiques de fréquentation). Un autre
objectif secondaire de ce travail est la recherche d’informations sur une identité, menant obligatoirement à
l’agglomération de ces données.

D’après ces critères, nous pouvons affirmer que l’utilisation de ce travail ne respecte pas les standards d’éthique
qu’un établissement comme la HEIG-VD veut s’imposer. Seuls les objectifs pédagogiques et d’évaluation doivent
être visés, et en aucun cas ce travail ne devra mener à la commercialisation d’un futur produit.

