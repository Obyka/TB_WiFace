\chapter{Cahier des charges}



\section*{Résumé du problème}
Les dispositifs Wifi diffusent en permanence des trames qui permettent de trouver rapidement les réseaux à
proximité. Ces trames, appelées “probe requests” sont utilisées par des “sniffers” pour “tracer” les utilisateurs dans
des centres commerciaux et d'autres endroits publics.

Ces informations jouissent pourtant d'un certain anonymat. En effet, les adresses MAC des dispositifs sont révélées
par ces trames mais il n'est normalement pas possible de les associer avec l'identité d'un individu. De plus, les
problématiques de « privacy » ont peu à peu amené les constructeurs à implémenter des mécanismes anonymisant
les utilisateurs, par exemple en rendant les adresses MAC pseudo-aléatoires.

Dans ce projet, l'étudiant ou l'étudiante devra concevoir et développer un démonstrateur pour un système capable
de combiner un “sniffer” amélioré capable de récolter les “probe request” et de prendre en même temps des
photos des visages se trouvant à proximité du capteur de trames (par exemple, en face d'une vitrine d'un magasin).
Le système tentera de corréler des adresses MAC et des images de visages récoltées à des endroits différents afin
d'associer un visage à une adresse. Finalement, une recherche par reconnaissance d'images sur les réseaux sociaux
(pour le démonstrateur, la recherche sera faite sur une base de données) essaiera d'obtenir l'identité du
propriétaire du téléphone mobile et toutes les données disponibles sur ce dernier. Des attaques visant la
désanonymisation pourront être mises en place afin de tracer au mieux les utilisateurs.

\clearpage
\subsection*{Objectifs du travail de Bachelor}
Au terme du travail de bachelor, les objectifs suivants auront été remplis :
\begin{itemize}
\item Le prototype développé permettra de scanner les probes requests à proximité et à en extraire les
informations utiles (adresse MAC, SSID)
\item Le prototype développé permettra de prendre des photos lorsqu’il détectera des visages dans son champs
d’action
\item Le prototype développé utilisera des mécanismes pour associer une identité, des photos, et des appareils
\item Le prototype développé utilisera des mécanismes de recherche d’information automatisée afin de
compléter les profils identifiés
\item L’architecture du projet permettra de faire fonctionner plusieurs prototypes en parallèle, en partageant les
mêmes données persistantes
\item Les données récoltées pendant le fonctionnement du prototype seront persistantes
\item Une étude sur la légalité et les enjeux éthiques de ce produit sera réalisée
\item Une étude sur les divers mécanismes d’anonymisation de l’adresse MAC, et l’état actuel d’implémentation
sera réalisée
\item Une étude sur la problématique de la reconnaissance faciale sans training set sera réalisée
\item À l’aide de la documentation produite et du matériel adéquat, le prototype devra être reproductible pour
un lecteur externe
\end{itemize}

Si le temps le permet, ainsi que les contraintes techniques, les objectifs suivants seront visés :
\begin{itemize}
\item Le prototype développé permettra de détecter la randomisation des adresses MAC
\item Le prototype développé utilisera des mécanismes actifs ou passifs afin d’attaquer la randomisation des
adresses MAC et ainsi de permettre la désanonymisation de l’utilisateur
\item Les différentes photos d’une personne pourront être regroupées sous la même identité à l’aide de
technologie de reconnaissance faciale, même sans « training set » initial
\end{itemize}


\subsection*{Déroulement global du travail}
Le travail peut être découpé en plusieurs parties, permettant une meilleure organisation générale du travail et des
délivrables :
\begin{enumerate}
\item Préparation au travail
\begin{itemize}
	\item Recherches initiales sur la problématiques et l’état de l’art
	\item S’informer sur les directives et le cadre imposé pour le travail de Bachelor
	\item Rédaction du présent cahier des charges
	\item Étude de marché pour le matériel nécessaire
\end{itemize}
\item Installation de l’environnement
\begin{itemize}
	\item Commande de matériel
	\item Installation de l’OS
	\item Installation de l’environnement de développement
\end{itemize}
\item Conception de la base de données
\begin{itemize}
	\item Modèle entité-association
	\item Modèle logique de données
	\item Script de création
\end{itemize}
\item Développement du scanner réseau
\begin{itemize}
	\item Capture des probes requests et Extraction des données
	\item (secondaire) Détecter la randomisation des adresses MAC
	\item (secondaire) Attaque de la randomisation des adresses MAC
	\item Insertion dans la base de données
	\item Tests du module
\end{itemize}
\item Développement du module de reconnaissance faciale
\begin{itemize}
	\item Prise de photo à la détection de visage
	\item Reconnaissance de visage
	\item Association probabiliste avec une ou plusieurs adresses MAC
	\item Insertion dans la base de données
	\item Tests du module
\end{itemize}
\item Test du prototype final
\item Documentation
\begin{itemize}
	\item Rédaction du cahier des charges
	\item Rédaction du journal de travail
	\item Rédaction du rapport intermédiaire et final
	\item Rédaction et recherche sur l’analyse légale et éthique
	\item Rédaction et recherche sur la partie théorique
	\item Rédaction du mode d’emploi
\end{itemize}
\end{enumerate}
\subsection*{Délivrables et résultats attendus}
Au terme du travail de bachelor, les délivrables suivants seront rendus :
\begin{enumerate}
\item Un rapport final qui contiendra, en plus du contenu imposé par les directives de la HEIG-VD :
	\begin{itemize}
	\item Une analyse sur la légalité et les enjeux éthiques de produits permettant l’identification et le
traçage des utilisateurs
	\item Une analyse sur le matériel à acquérir pour développer le prototype
	\item La description de chaque étape d’implémentation
	\item Une partie théorique sur au moins un des aspects suivants (reconnaissance faciale, identification
d’un appareil à l’aide de probe request, attaques sur les mécanismes de protection de l’identité)
	\item Un mode d’emploi permettant l’installation et l’utilisation du prototype
	\end{itemize}
\item Un prototype remplissant les exigences mentionnées plus haut, utilisable pour une démonstration
\end{enumerate}

Avant le 19 juin 2020, un rapport intermédiaire sera rendu.



