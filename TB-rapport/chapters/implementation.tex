\chapter{Implémentation}
\label{ch:implémentation}

\section{La base de données}

\section{L'API WiFace}

\subsection{Choix de la stack}

Du côté serveur, le langage utilisé est le Python (3.7). Pour faciliter le développement d’une API rest, le micro-
framework \textbf{Flask} a été choisi. Au vu de la documentation et de mon expérience personnelle, ce choix est pertinent
dans le cadre de ce projet.

Avantages de Flask :
\begin{itemize}
\item Simple et léger
\item Convient bien au développement d’application de petite ou moyenne envergure
\item Très flexible
\item Prise en main rapide
\item Compatible avec l’ORM sqlalchemy
\end{itemize}

L’ORM qui a été choisi pour fonctionner avec Flask est \textbf{SQLAlchemy}. Il existe en effet un module python flask-
sqlalchemy qui rend l’intégration simple.
Les données sont sérialisées à l’aide de \textbf{marshmallow}.
La gestion de l’authentification se fait à l’aide de token JWT et du module correspondant \textbf{flask-jwt} et \textbf{pyjwt}
La spécification est écrite à l’aide de swagger, qui propose également une documentation automatique.

\section{Le client Raspberry}

\section{Algorithme PP2I}