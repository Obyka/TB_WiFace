\chapter{Guides d'installation}
\label{ch:guide_installation}

Ce chapitre traitera de l'installation de tous les composants de l'architecture comprenant notamment: 
\begin{itemize}
    \item Le serveur Flask
    \item La base de données SQLite partiellement peuplée
    \item Le(s) client(s) Raspberry Pi avec leurs modules (PiCaméra et antenne 802.11)
\end{itemize}
Toutes ces étapes seront séparées en deux parties: un guide pour le client, et un guide pour le serveur 
\section{Installation du serveur WiFace}
Pour rappel, le serveur WiFace ou "WiFace API" (cf \ref{fig:diag_archi}) endosse les responsabilités
suivantes: 
\begin{itemize}
    \item Récolte, persiste, et traite les données envoyées par les clients (Raspberry ou utilisateur front-end)
    \item Fourni de l'authentification
    \item Offre une API
    \item Offre une interface web pour les opérateurs humains
    \item Interagit avec les systèmes externes nécessaire au traitement des données (e.g AWS Rekognition)
\end{itemize}

\subsection{Prérequis software}
\begin{itemize}
    \item Docker : Le serveur est dockerisé afin de faciliter l'installation
    \item git: Les fichiers nécessaires se trouve sur github
\end{itemize}

\subsection{Marche à suivre}
La marche à suivre pour exécuter le serveur est la suivante:
\begin{enumerate}
    \item Clôner le repository \textbf{TB WiFace}
    \begin{listingsbox}{console}{Clônage du dépôt sur github}
git clone https://github.com/Obyka/TB_WiFace.git
    \end{listingsbox}
    \item{Se déplacer dans le sous-dossier API}
    \item{Configurer le script de démarrage start.sh}
\end{enumerate}
    Il existe deux modes pour la création de la base de donnée. Le premier
    consiste à inclure quatres identités inscrites dans le script build\_database.py.
    Le deuxième consiste à lancer une simulation pour l'algorithme PP2I et d'y inscrire les résultats.
    Les paramètres de la simulation peuvent être ajusté dans le script start.sh
    
\begin{listingsbox}{console}{Exemple de configuration de start.sh}
#!/bin/bash
simulation=0
nb_person=10
duration=1000

if [[ $simulation -eq 1 ]];
then
    python build_database.py --nb_person $nb_person \
    --duration $duration --simulation && python3 server.py
else
    python3 build_database.py && python3 server.py
fi    
\end{listingsbox}
    
\begin{enumerate}[resume]
    \item Modifier le fichier API/config.py. Plusieurs constantes relatives à la sécurité doivent être modifiées (credentials Amazon, JWT et CSRF secrets)
    \item Modifier l'utilisateur par défaut dans le fichier API/build\_database.py et s'assurer qu'il soit bien admin.
\end{enumerate}

\begin{enumerate}[resume]
    \item Construire l'image docker correspondante et la démarrer.
\end{enumerate}

Le build peut prendre quelques minutes afin d'obtenir les dépendances python.
Ici, le port 5555 sera exposé.
\begin{listingsbox}{console}{Création et démarrage de l'image docker}
docker build -t wiface-api .
docker run -p 5555:5000 wiface-api
\end{listingsbox}

Le serveur est maintenant disponible aus adresses api/ et web/.
