\chapter{Guides d'installation}
\label{ch:guide_installation}

Ce chapitre traitera de l'installation de tous les composants de l'architecture comprenant notamment: 
\begin{itemize}
    \item Le serveur Flask
    \item La base de données SQLite partiellement peuplée
    \item Le(s) client(s) Raspberry Pi avec leurs modules (PiCaméra et antenne 802.11)
\end{itemize}
Toutes ces étapes seront séparées en deux parties: un guide pour le client, et un guide pour le serveur 
\section{Installation du serveur WiFace}
Pour rappel, le serveur WiFace ou "WiFace API" (cf \ref{fig:diag_archi}) endosse les responsabilités
suivantes: 
\begin{itemize}
    \item Récolte, persiste, et traite les données envoyées par les clients (Raspberry ou utilisateur front-end)
    \item Fourni de l'authentification
    \item Offre une API
    \item Offre une interface web pour les opérateurs humains
    \item Interagit avec les systèmes externes nécessaire au traitement des données (e.g AWS Rekognition)
\end{itemize}

\subsection{Prérequis software}
\begin{itemize}
    \item Docker : Le serveur est dockerisé afin de faciliter l'installation
    \item git: Les fichiers nécessaires se trouve sur github
\end{itemize}

\subsection{Marche à suivre}
La marche à suivre pour exécuter le serveur est la suivante:
\begin{enumerate}
    \item Clôner le repository \textbf{TB WiFace}
    \begin{listingsbox}{console}{Clônage du dépôt sur github}
git clone https://github.com/Obyka/TB_WiFace.git
    \end{listingsbox}
    \item{Se déplacer dans le sous-dossier API}
    \item{Configurer le script de démarrage start.sh}
\end{enumerate}
    Il existe deux modes pour la création de la base de donnée. Le premier
    consiste à inclure quatres identités inscrites dans le script build\_database.py.
    Le deuxième consiste à lancer une simulation pour l'algorithme PP2I et d'y inscrire les résultats.
    Les paramètres de la simulation peuvent être ajusté dans le script start.sh
    
\begin{listingsbox}{console}{Exemple de configuration de start.sh}
#!/bin/bash
simulation=0
nb_person=10
duration=1000

if [[ $simulation -eq 1 ]];
then
    python build_database.py --nb_person $nb_person \
    --duration $duration --simulation && python3 server.py
else
    python3 build_database.py && python3 server.py
fi    
\end{listingsbox}
    
\begin{enumerate}[resume]
    \item Modifier le fichier API/config.py. Plusieurs constantes relatives à la sécurité doivent être modifiées (credentials Amazon, JWT et CSRF secrets)
    \item Modifier l'utilisateur par défaut dans le fichier API/build\_database.py et s'assurer qu'il soit bien admin.
\end{enumerate}

\begin{enumerate}[resume]
    \item Construire l'image docker correspondante et la démarrer.
\end{enumerate}

Le build peut prendre quelques minutes afin d'obtenir les dépendances python.
Ici, le port 5555 sera exposé.
\begin{listingsbox}{console}{Création et démarrage de l'image docker}
docker build -t wiface-api .
docker run -p 5555:5000 wiface-api
\end{listingsbox}

Le serveur est maintenant disponible aus adresses api/ et web/.

\section{Installation d'un client Raspberry Pi}
Cette section détaille l'installation, la configuration et l'utilisation d'un client Raspberry Pi.

\subsection{Prérequis}

\subsubsection{Prérequis hardware}
\begin{itemize}
    \item Une Raspberry Pi 4 Model B avec 4GB de RAM
    \item Un module Pi Camera v2
    \item Une antenne 802.11 compatible avec le mode Monitor
\end{itemize}

\subsubsection{Prérequis software}
\begin{itemize}
    \item Raspbian 10 (Buster)
    \item Python 3.7.3
\end{itemize}

\subsection{Création du compte local}
Afin de réduire la surface d'attaque, un compte va être créé pour ne pas utiliser
le profil par défaut (pi).
La commande sudo sera aussi configurée pour demander systèmatiquement un mot de passe.
Ce compte sera ensuite ajouté au groupe sudo, ce qui est nécessaire pour utiliser Scapy.
Pour des raisons pratiques, le mot de passe utilisé sera inscrit dans ce rapport. En cas d'utilisation de WiFace, merci de changer ce dernier.

Le compte par défaut ne devrait pas être supprimé, pour des raisons
de dépendances. Son mot de passe doit toutefois être modifié.

\begin{listingsbox}{console}{Création d'un compte local}
# Changer le mot de passe de l'utilisateur pi
passwd

sudo adduser wiface-user
New password: Mu7%0&J#0!X^&t6OHHQ@z&
sudo usermod -a -G adm,dialout,cdrom,sudo,audio\
,video,plugdev,games,users,input,netdev,gpio,i2c,spi wiface-user

# Changer les lignes correspondantes à Pi et wiface-user par 
# pi ALL=(ALL) PASSWD: ALL
sudo visudo /etc/sudoers.d/010_pi-nopasswd
\end{listingsbox}

Pour faciliter la gestion du client, une paire de clé ssh va être créée pour permettre la connexion distante.
Pour cela, il faut activer le serveur ssh sur le client et copier la clé publique dessus.

\begin{listingsbox}{console}{Ajout d'une clé SSH}
    # Activer SSH sous "Interfacing options"
    sudo raspi-config

    # Sur le PC qui va se connecter au client
    ssh-keygen
    ssh-copy-id wiface-user@<IP-ADDRESS>
    ssh wiface-user@<IP-ADDRESS>

    # Il faut désactiver la connexion par mot de passe pour
    # profiter de la sécurité offerte par les clés ssh.
    sudo nano /etc/ssh/sshd_config
    ChallengeResponseAuthentication no
    PasswordAuthentication no
    UsePAM no

    # Relancer le service
    sudo service ssh reload
\end{listingsbox}

\subsection{Installation des dépendances}
Maintenant que la configuration est faite, il faut installer les dépendances
nécessaires à l'exécution du client. 

\begin{listingsbox}{console}{Installation des dépendances du client}
    sudo pip3 install --pre scapy[basic]
    
\end{listingsbox}


\subsection{Installation de l'antenne 802.11}
Pour faire fonctionner l'antenne en mode "Monitor", nous allons utiliser Aircrack-ng.\footnote{Aircrack-ng est une suite de logiciel permettant des opérations sur les périphériques, protocoles, et réseau WiFi.}
\begin{listingsbox}{console}{Installation de aircrack-ng}
sudo apt-get update
sudo apt-get upgrade
sudo apt-get install aircrack-ng

# Si le programme a été installé correctement, la commande
# devrait retourner les différentes interfaces réseaux.
sudo airmon-ng
\end{listingsbox}

Une fois l'antenne connectée, la dernière commande devrait retourner une ligne supplémentaire. 
Cela veut dire que le périphérique a bien été détecté, nous pouvons donc le passer en mode Monitor.

\begin{listingsbox}{console}{Passage de l'interface en mode Monitor}
sudo airmon-ng start <Interface-Name>
\end{listingsbox}

L'interface est maintenant prête pour le scan.
    


